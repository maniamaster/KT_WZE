\NeedsTeXFormat{LaTeX2e}[2005/12/01]
%%    2010/04/06 v1.0 Vorlage Master-Forschungspraktikum Versuchsauswertung
%%    based on the 2009/10/14 v0.1 GAUBM template by Prof Pruschke

\documentclass[twoside,        %% zweiseitiges Layout
               BCOR12mm,       %% Bindekorrektur 12 mm
% please comment out if report is in English
%               english,ngerman, %% Dokumentspr. Deutsch, Alternativspr. Englisch
% please remove comment if report is in English 
               ngerman,english, %% Dokumentspr. Englisch, Alternativspr. Deutsch
               fleqn,headsepline=false,footsepline=false
              ]{Vorlage/MFPREPORT}
\makeatletter
\DeclareOldFontCommand{\rm}{\normalfont\rmfamily}{\mathrm}
\DeclareOldFontCommand{\sf}{\normalfont\sffamily}{\mathsf}
\DeclareOldFontCommand{\tt}{\normalfont\ttfamily}{\mathtt}
\DeclareOldFontCommand{\bf}{\normalfont\bfseries}{\mathbf}
\DeclareOldFontCommand{\it}{\normalfont\itshape}{\mathit}
\DeclareOldFontCommand{\sl}{\normalfont\slshape}{\@nomath\sl}
\DeclareOldFontCommand{\sc}{\normalfont\scshape}{\@nomath\sc}
\makeatother

%% Pakete und Definitionen ausgelagert
\usepackage{a4}
\usepackage{multicol}

% language option set in JGNSUM class
\usepackage{babel}
\usepackage{hyperref}

%% FONT:
%\usepackage{lmodern}
\usepackage{times} % sieht besser aus als lmodern
%\usepackage{palatino} % sieht schlechter aus als times
%\usepackage{mathpazo} % very ugly font, to be loaded later ???
%\usepackage{cmbright} % doesn't work either
\usepackage[T1]{fontenc}
\usepackage{textcomp}

\usepackage{ucs}
\usepackage[utf8x]{inputenc}

\usepackage{amsfonts}
\usepackage{amstext}
\usepackage{amsmath}
\usepackage{amsthm}
\usepackage{amssymb}
\usepackage{amsbsy}   % AMS-Boldsymbol

% \usepackage{mathabx} % e.g. for \Sun
%% but not a standard package (neither texlive nor Miktex)
%% so use wasysym (\astrosun) instead
\usepackage{wasysym} % e.g. for \astrosun or \CheckedBox

\usepackage{bbm,mathrsfs}

\usepackage{textcomp} % noch einige coole symbole

\usepackage{sectsty}
\allsectionsfont{\raggedright}

\usepackage[numbers]{natbib}
\citestyle{dinat}
\bibliographystyle{dinat}

\usepackage{makeidx}

\usepackage{url}	% für hübsche URLs mit Link
\usepackage{color}	% für farben a la \definecolor{Gray}{gray}{0.5}
\usepackage{verbatim}
\usepackage{subfigure}
\usepackage{listings}

\usepackage{fancybox}
%usage:
%\begin{Verbatim}[frame=single,label=Titel]
%Verbatim Zeile
%\end{Verbatim}


 \setlength{\textwidth}{16.2cm}
 \setlength{\textheight}{24cm}
 \setlength{\oddsidemargin}{0cm}
 \setlength{\evensidemargin}{-0.5cm}

 %unbedingt nach abmessungen einfügen!
 \usepackage{fancyhdr}
 \pagestyle{fancy}
 %\sloppy % für weniger absatzfehler

 \setcounter{tocdepth}{2}
 \setcounter{secnumdepth}{2}

 \ifreportelse{\numberwithin{equation}{chapter}}{\numberwithin{equation}{section}}
 \theoremstyle{plain}% default
 \ifreportelse{\newtheorem{thm}{Theorem}[chapter]}{\newtheorem{thm}{Theorem}[section]}

 \newtheorem{satz}{Satz}
 \newtheorem{lem}[thm]{Lemma}
 \newtheorem{prop}[thm]{Proposition}
 \newtheorem{kor}[thm]{Korollar}
 \newtheorem{cor}[thm]{Corollary}

 \theoremstyle{definition}
 \newtheorem{defi}{Definition}

 \def\@proof{%
  \if@englishpreamble{Proof}\else{Beweis}\fi
 }
 \newenvironment{bew}{\begin{proof}[\@proof]}{\end{proof}}



%% einbinden einiger nützlicher Befehle
\newcommand{\iflanggerman}[2]{
 \iflanguage{german}{#1}{
  \iflanguage{ngerman}{#1}{#2}
 }
}

% box around the whole equation, number inclusive
\newcommand{\boxedeqn}[1]{%
  \[\fbox{%
      \addtolength{\linewidth}{-2\fboxsep}%
      \addtolength{\linewidth}{-2\fboxrule}%
      \begin{minipage}{\linewidth}%
      \begin{equation}#1\end{equation}%
      \end{minipage}%
    }\]%
}

\iflanggerman{
 \newcommand{\const}{\mathrm{konst}}
 \newcommand{\Const}{\mathrm{konst.}}
}{
 \newcommand{\const}{\mathrm{const}}
 \newcommand{\Const}{\mathrm{const.}}
}

% von Meier
\newcommand{\nbd}{\nobreakdash-\hspace{0pt}}
% example: $K$\nbd{}Vektorraum
\newcommand*{\transpose}[1]{\prescript{t}{}{#1}}
\newcommand*{\conjugate}[1]{\overline{#1}}
\newcommand*{\abs}[1]{\lvert#1\rvert}
\newcommand*{\Mod}{\mathrm{mod}}
\newcommand{\symdif}{\mathbin\triangle}
\DeclareMathOperator{\Graph}{Graph}
\DeclareMathOperator{\id}{id}
\DeclareMathOperator*{\grad}{grad}
\DeclareMathOperator*{\Div}{div}
\DeclareMathOperator*{\rot}{rot}
\DeclareMathOperator{\sig}{sig}
\DeclareMathOperator{\sgn}{sgn}
\DeclareMathOperator{\diag}{diag}
\DeclareMathOperator{\tr}{tr}
\DeclareMathOperator{\Sp}{Sp}
\DeclareMathOperator{\im}{Im}
\DeclareMathOperator{\re}{Re}

\newcommand{\vcentcolon}{\mathop{:}}



%Zur Formatierung in der Matheumgebung
\renewcommand{\t}{\ensuremath{\rm\tiny}} % Tiefgestellter Text in der Matheumgebung wird schoener mit: $\Phi_{\t{Text}}$
\renewcommand{\d}{\ensuremath{\mathrm{d}}} % Die totale Ableitung ist stets aufrecht zu setzen: \d
\newcommand{\diff}[3][]{\ensuremath{\frac{\d^{#1}#2}{\d#3^{#1}}}} % einfache Ableitung nach x: $\ddx{\Phi}$
\newcommand{\pdiff}[3][]{\ensuremath{\frac{\partial^{#1}#2}{\partial#3^{#1}}}} % wie gesprochen, eine partielle Ableitung: \del
\newcommand{\aeqiv}{\ensuremath{\qquad \Longleftrightarrow \qquad}} % Eine Aequivalenz
\newcommand{\folgt}{\ensuremath{\qquad \Longrightarrow \qquad}} % Ein Folgepfeil mit Abstaenden
\newcommand{\corresponds}{\ensuremath{\mathrel{\widehat{=}}}} % Befehl für "Entspricht"-Zeichen
\newcommand{\mi}[1]{\ensuremath{\mathit{#1}}} % italics für griechische Buchstaben in Matheumgebung

%Um nicht so viel schreiben zu müssen...
\newcommand{\bs}[1]{\boldsymbol{#1}}
\newcommand{\ol}[1]{\overline{#1}}
\newcommand{\wtilde}[1]{\widetilde{#1}}
\newcommand{\mrm}[1]{\mathrm{#1}}
\newcommand{\mbf}[1]{\mathbf{#1}}
\newcommand{\mbb}[1]{\mathbb{#1}}
\newcommand{\mcal}[1]{\mathcal{#1}}
\newcommand{\mfrak}[1]{\mathfrak{#1}}

%Abkürzungen
\newcommand{\zB}{z.\,B.\ }
\newcommand{\bzw}{b.\,z.\, w.\ }
\newcommand{\Dh}{d.\,h.\ }
\newcommand{\Gl}{Gl.\ }
\newcommand{\Abb}{Abb.\ }
\newcommand{\Tab}{Tab.\ }

\usepackage{braket}


\begin{document}
\LabratoryName{KT.WZE}{W/Z experiment at the Tevatron}
\ProtocolAuthor{Eric}{Bertok}{eric.bertok@stud.uni-goettingen.de}
\Assistant{Dr. J. Veatch}{}
\ResearchFocus{Nuclear and particle physics (M.phy.404)}
\ConductedOn{24}{01}{2018}
\date{\today}
% eines von beiden
\CopyNotWanted
%\CopyWanted

\pagenumbering{roman}
\maketitle

%\begin{otherlanguage}{english}
%\end{otherlanguage}

\tableofcontents

\clearpage
\pagenumbering{arabic}

\section{Introduction}
\label{sec:introduction}
In goal of this experiment is the determination the  of the branching ratio of the $W$ boson
BR($W\rightarrow\mu\nu$). First, $W$ and $Z$ bosons are reconstructed using
data provided by the Tevatron collider at Fermilab. By comparing with Monte Carlo
simulations, selection parameters are obtained, which allow for clean cuts for
filtering out background events (jets and cosmic source). The mass and the
transverse mass is then determined for the $Z$ and $W$ boson respectively.
Finally the branching ratio is calculated from the number of selected events,
the trigger efficiencies, as well as the reconstruction efficiencies.


\section{Theory}
\label{sec:theory}
\subsection{Electroweak interaction}
The GWS theory (Glashow, Weinberg, Salam) is the unified description of both
the electromagnetic force mediated by the photon and the weak interaction
mediated by the massive $W^+,W^-$ and the neutral neutral $Z$ boson. It was
confirmed experimentally in the 1970s \cite{wikigsw}. 
The gauge bosons are introduced by means of a local SU(2)$_L$ gauge symmetry in
a weak isospin space. The weak isospin doublets are formed by fermions
differing by one unit of charge \cite[p.;416]{thomson}. By also replacing the
U(1) symmetry by a new U(1)$_Y$ symmetry with the ``hypercharge'' $Y$, the
neutral $Z$ boson can be identified by a linear combination of the neutral
$W^{(3)}$ boson and the $B$ boson coupling to the hypercharge. More details can
be found in \cite[p.\;418ff]{thomson}.
Being a charged boson,
the $W$ bosons couple to fermions differing by one unit of charge. Furthermore
it maximally violates parity as it only couples to left-handed particles and
right-handed antiparticles. The vertex factor is given by \cite[p.;409]{thomson}
\begin{align}
    \label{eq:vertexw}
    -i\frac{g_W}{\sqrt{2}}\frac{1}{2}\gamma^\mu(1-\gamma^5),
\end{align} where $g_W$ is the weak coupling constant and $\gamma^\mu$ are the
gamma matrices. The $Z$ boson however, couples to any pair of identical
fermions, albeit coupling more strongly to left handed ones. This becomes
apparent in the form of the vertex factor: \cite[p.\;432]{thomson}
\begin{align}
    \label{eq:vertexz}
    -i\frac{1}{2}g_Z \gamma^\mu(c_V-c_A\gamma^5),
\end{align}
with the vector and axial vector couplings $c_V$ and $c_A$.

\subsection{Matrix elements and Decay rates}
The matrix elements for the electroweak interaction can be calculated with the
appropriate Feynman rules.
After averaging over the three possible polarizations, the spin-averaged matrix
element squares is obtained for both the $W$ and the $Z$ boson decaying to a
lepton and its neutrino or a lepton- anti-lepton pair, respectively
\cite[p.;242,411]{thomson}:
\begin{align}
    \label{eq:spinavg}
    \braket{|\mathcal{M}_W^2|}&=\frac{1}{3}g_W^2m_W^2\\
    \braket{|\mathcal{M}_Z^2|}&=\frac{1}{3}(c_V^2+c_A^2)g_Z^2m_Z^2.
\end{align}
These can be inserted into the decay rate formula: \cite[p.\;411]{thomson}
\begin{align}
    \label{eq:decayrate}
    \Gamma=\frac{p^*}{32\pi^2m^2}\int
    \braket{|\mathcal{M}^2|}\d\Omega=\frac{p^*}{8\pi m^2}
    \braket{|\mathcal{M}^2|},
\end{align}
where $m$ is the mass of the boson and $p*$ is the momentum of the lepton in
the center of mass frame.
One can argue that $p*=m_Z/2$, as the
decay happens in the centre of mass frame of the decaying particle.
Therefore the decay rate is 
\begin{align}
    \label{eq:decay}
    \Gamma(W^-\rightarrow e^- \bar{\nu}_e)&=\frac{g_W^2m_W}{48\pi}.\\
    \Gamma(Z\rightarrow e^- e^+)&=\frac{g_Z^2m_Z}{48\pi}(c_V^2+c_A^2).
\end{align}
Lepton universality tells us that this is the same for all three leptonic
channels when neglecting masses. For hadronic processes, the CKM matrix has to
be considered, while excluding the top quark, as it is too massive.
For the $W$ boson, one obtains for the decay width [cite]
\begin{align}
\Gamma_W=(3+6 \kappa)\Gamma(W^-\rightarrow e^-
\bar\nu_e)\approx9.2\;\frac{g_W^2m_W}{48\pi}=2.1\;\text{GeV.}
    \label{eq:gammaw}
\end{align}
$\kappa\approx1.038$ is a correction factor that accounts for second order QCD processes.
Similarly, for the $Z$ boson, one obtains
\begin{align}
    \Gamma_Z\approx2.5\;\text{GeV.}    \label{eq:gammaZ}
\end{align}
The branching ratios for the muon channel are therefore
\begin{align}
    BR(W\rightarrow\mu\bar\nu_\mu)&=10.8\%,\\
    BR(Z\rightarrow\mu^+\mu^-)&=3.5\%.
    \label{eq:branches}
\end{align}
\subsection{Invariant and transverse mass}
For the $Z$ boson one can calculate the functional form of the invariant mass
peak by taking into account its finite lifetime.
The cross section for a $q\bar q\rightarrow \mu^+\mu^-$ event is proportional
to [cite]
\begin{align}
    \sigma\propto|\mathcal{M}|^2\propto\left|\frac{1}{q^2-m_Z^2+im_Z\gamma_Z}\right|^2=\frac{1}{(q^2-m_Z^2)^2+m_Z^2\gamma_Z^2},
    \label{eq:breit}
\end{align}which is a Breit-Wigner curve.
$q$ is the invariant mass of both muons. As both can be detected in such an
event, the Breit-Wigner-curve can be fitted directly to the selected data to
obtain the mass of the $Z$ boson.
For the $W$ boson, things are more complicated. Due to the $W$ events only
having one muon, the undetectable neutrino has to be reconstructed from the
missing momentum. For a hadronic collider such as the tevatron, the total
centre of mass energy cannot be known on an event to event basis [cite] due to
the composite nature of the hadrons. More specifically, the $z$-momentum of the
interacting partons are unknown, making the invariant mass reconstruction
impossible. However, one can define the transverse mass $M_T$, which can be
calculated from the reconstructed transverse momentum of the neutrino
$\vec{p}_T^\mu$. First, the missing transverse energy $MET$ is determined as
\begin{align}
    MET\approx|\bold{p}_T^\nu=|-\bold{p}_T^\mu-\bold{u}_T,
    \label{eq:met}
\end{align}where $\vec{u}_T$ is the transverse momentum of the hadrons [cite].
The transverse mass is then defined as
\begin{align}
    M_T=\sqrt{(MET+\bold{p}_T^\mu)^2-(MET_x+p_x^\mu)^2-(MET_y+p_y^\mu)^2}.
    \label{eq:mt}
\end{align}
This quantity is lorantz invariant but does not peak at $M_Z$. However, the $W$
mass can be read off from the position of the dropoff, as the longitudinal
component of the invariant mass is then close to zero [cite].


\section{Execution}
\label{sec:execution}
\section{Analysis}
\label{sec:analysis}
\section{Discussion}

\bibliography{literatur}
\end{document}

